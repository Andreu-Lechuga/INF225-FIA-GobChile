\documentclass[12pt]{article}
\usepackage[utf8]{inputenc}
\usepackage[spanish]{babel}
\usepackage{graphicx}
\usepackage{geometry}
\usepackage{fancyhdr}
\usepackage{parskip}
\usepackage{titlesec}
\usepackage{setspace}
\usepackage{xcolor}
\usepackage{lipsum}
\usepackage{changepage}



% Márgenes y estilo
\geometry{a4paper, margin=2.5cm}
\setstretch{1.15}

% Encabezado y pie de página
\setlength{\headheight}{15pt}
\pagestyle{fancy}
\fancyhf{}
\rhead{Hito 1: SEMAT}
\lhead{INF255 - Ingeniería de Software}
\cfoot{\thepage}

% Título de secciones en negrita y tamaño ajustado
\titleformat{\section}{\large\bfseries}{\thesection}{1em}{}

% estilo personalizado section
\titleformat{\section}
  {\normalfont\Large\bfseries} % formato
  {\thesection.} % número
  {0.5em} % espacio entre número y título
  {} % antes del título

% comandos personalizados
\newcommand{\bloqueind}[1]{%
  \noindent\hspace*{2cm}%
  \parbox{\dimexpr\linewidth-2cm}{#1}%
}


\begin{document}

% LOGO y encabezado
\begin{minipage}{0.2\textwidth}
    \includegraphics[width=\linewidth]{logo-usm.jpg} % Cambia logo.png por tu archivo real
\end{minipage}
\hfill
\begin{minipage}{0.75\textwidth}
    \centering
    \vspace{0.5cm}
    {\LARGE \textbf{Hito 1: SEMAT}}\\[0.4cm]
    {\large Universidad Tecnica Federico Santa Maria}\\
    {\large Asignatura: INF225 - Ingeniería de Software}\\
    {\large Fecha: \today}
\end{minipage}

\vspace{1.5cm}

\begin{center}
    \textbf{Grupo 1} \\
    \vspace{0.5em}
    \begin{tabular}{r l}
        Matias Pajarito Catalan & \textbar \quad 202273522-8 \\
        Pedro Miranda Miranda   & \textbar \quad 201930556-5 \\
        Aymara Rojas Arellano   & \textbar \quad 202004665-4 \\
        Andreu Lechuga Gonzalez & \textbar \quad 202073595-6 \\
    \end{tabular}
\end{center}

\vspace{1cm}

\section{Introducción}

El presente documento tiene como propósito analizar
el desarrollo de nuestro proyecto de software utilizando
el \textbf{framework SEMAT (Software Engineering Method
    and Theory)}. Este enfoque nos permite observar el
progreso y madurez del proyecto desde distintas
perspectivas, conocidas como \textbf{Alphas}, facilitando una
gestión más clara, estructurada y alineada con buenas
prácticas de ingeniería de software.


\section{Autoevaluacion SEMAT: Estados Actuales}


%%%%%%%%%%%% Alpha 1
\subsection{Alpha 1: Requisitos}


\textbf{Estado Actual:} \textit{Acotado} \\
\hspace*{1cm} Los requisitos están definidos, pero falta validación con usuarios.

\textbf{Próximo Estado Deseado:} \textit{Coherente} \\
\hspace*{1cm} Los requisitos están alineados con las necesidades reales.


\vspace{0.1cm}

\textbf{Acciones Necesarias}
\begin{itemize}
    \item Validacion con \textbf{Usuarios Reales:} Verificar que tanto
          el boletín como el Software cumpla con las expectativas del cliente.
    \item Revision de \textbf{Requisitos:} Eliminar ambigüedades en los
          requisitos actuales y definir criterios de aceptación nitidos y alcanzables.
\end{itemize}

\vspace{0.1cm}

\textbf{Justificación:} Se han definido los requisitos, pero falta validación de usuarios finales.

\vspace{0.3cm}

%%%%%%%%%% Alpha 2
\subsection{Alpha 2: Sistema de Software}


\textbf{Estado Actual:} \textit{Demostrable} \\
\hspace*{1cm} El sistema es capaz de generar boletines, sin embargo aun
no es completamente estable ni esta correctamente optimizado \\
\textbf{Próximo Estado Deseado:} \textit{Usable} \\
\hspace*{1cm}

\vspace{0.1cm}

\textbf{Acciones Necesarias}


\vspace{0.1cm}

\textbf{Justificación:}

\vspace{0.3cm}

%%%%%%%%%%% Alpha 3
\subsection{Alpha 3: Equipo}


\textbf{Estado Actual:} \textit{Coordinado} \\
\hspace*{1cm} Los roles del equipo estan bien definidos
pero falta mejorar la eficiencia en la
comunicacion y en los procesos de trabajo. \\
\textbf{Próximo Estado Deseado:} \textit{Colaborando} \\
\hspace*{1cm} Flujo de trabajo optimizado y equipo en
sincronizacion.

\vspace{0.1cm}

\textbf{Acciones Necesarias}
\begin{itemize}
    \item Mejorar nuestra \textbf{Planificacion y Seguimiento de Tareas},
          utilizando herramientas como Jira, Trello o GitHub Projects.
    \item Definir \textbf{Reuniones Periodicas de Revision}
          para evaluar avances y problemas que vayan surgiendo
          durante nuestro proceso de desarrollo.
\end{itemize}

\vspace{0.1cm}

\textbf{Justificación:} El equipo tiene
roles definidos, pero falta mayor
eficiencia en comunicación y colaboración.

\vspace{0.3cm}

%%%%%%%%%%%%%%%%%%%%%%%%%%%%%% Alpha 4
\subsection{Alpha 4: Trabajo}

\textbf{Estado Actual:} \textit{En Progreso} \\
\hspace*{1cm} Las tareas principales estan
avanzando, pero hay bloqueo en otras tareas
de menor prioridad \\
\textbf{Próximo Estado Deseado:} \textit{Cumplido} \\
\hspace*{1cm} El flujo de desarrollo es
eficiente, enfocado y libre de bloqueos

\vspace{0.1cm}

\textbf{Acciones Necesarias}
\begin{itemize}
    \item Reducir \textbf{Dependencias de APIs Externas}
          para evitar bloqueos en la generacion
          de boletines.
    \item Identificar \textbf{Cuellos de Botella}
          en el flujo de trabajo y optimizar nuestros
          procesos de desarrollo.
\end{itemize}

\vspace{0.1cm}

\textbf{Justificación:} Se estan realizando
avances pero todavia existen bloques por
dependencias o falta de optimizacion.

\vspace{0.3cm}

\subsection{Alpha 5: Stakeholders}

\textbf{Estado Actual:} \textit{} \\
\hspace*{1cm} \\
\textbf{Próximo Estado Deseado:} \textit{} \\
\hspace*{1cm}

\vspace{0.1cm}

\textbf{Acciones Necesarias}
\begin{itemize}
    \item
    \item
\end{itemize}

\vspace{0.1cm}

\textbf{Justificación:}

\vspace{0.3cm}


%\subsection{}
%
%\textbf{Estado Actual:} \textit{} \\
%\hspace*{1cm} \\
%\textbf{Próximo Estado Deseado:} \textit{} \\
%\hspace*{1cm}
%
%\vspace{0.1cm}
%
%\textbf{Acciones Necesarias}
%\begin{itemize}
%    \item 
%    \item 
%\end{itemize}
%
%\vspace{0.1cm}
%
%\textbf{Justificación:} 
%
%\vspace{0.3cm}



\end{document}