\documentclass[12pt]{article}
\usepackage[utf8]{inputenc}
\usepackage[spanish]{babel}
\usepackage{graphicx}
\usepackage{geometry}
\usepackage{fancyhdr}
\usepackage{parskip}
\usepackage{titlesec}
\usepackage{setspace}

% Márgenes y estilo
\geometry{a4paper, margin=2.5cm}
\setstretch{1.15}

% Encabezado y pie de página
\setlength{\headheight}{15pt}
\pagestyle{fancy}
\fancyhf{}
\rhead{Hito 1: SEMAT}
\lhead{INF225 - Ingeniería de Software}
\cfoot{\thepage}

% Título de secciones en negrita y tamaño ajustado
\titleformat{\section}{\large\bfseries}{\thesection}{1em}{}

\begin{document}

% LOGO y encabezado
\begin{minipage}{0.2\textwidth}
    \includegraphics[width=\linewidth]{logo-usm.jpg} % Cambia logo.png por tu archivo real
\end{minipage}
\hfill
\begin{minipage}{0.75\textwidth}
    \centering
    \vspace{0.5cm}
    {\LARGE \textbf{Hito 1: SEMAT}}\\[0.4cm]
    {\large Universidad Tecnica Federico Santa Maria}\\
    {\large Asignatura: INF225 - Ingeniería de Software}\\
    {\large Fecha: \today}
\end{minipage}

\vspace{1.5cm}

% Participantes
\noindent
\textbf{Grupo 1:}\\
Matias Pajarito Catalan - \\
Pedro Miranda Miranda - \\
Aymara Rojas Arellano - \\
Andreu Lechuga Gonzalez - 202073595-6

\vspace{1cm}

\section*{Introducción}

El presente documento tiene como propósito analizar 
el desarrollo de nuestro proyecto de software utilizando 
el \textbf{framework SEMAT (Software Engineering Method 
and Theory)}. Este enfoque nos permite observar el 
progreso y madurez del proyecto desde distintas 
perspectivas, conocidas como "Alphas", facilitando una 
gestión más clara, estructurada y alineada con buenas 
prácticas de ingeniería de software.

\newpage

\section{Alpha 1 supongo}
estoy pensando como continuar

\end{document}