\documentclass[12pt]{article}
\usepackage[utf8]{inputenc}
\usepackage[spanish]{babel}
\usepackage{graphicx}
\usepackage{geometry}
\usepackage{fancyhdr}
\usepackage{parskip}
\usepackage{titlesec}
\usepackage{setspace}

% Márgenes y estilo
\geometry{a4paper, margin=2.5cm}
\setstretch{1.15}

% Encabezado y pie de página
\setlength{\headheight}{15pt}
\pagestyle{fancy}
\fancyhf{}
\rhead{Hito 1: SEMAT}
\lhead{INF225 - Ingeniería de Software}
\cfoot{\thepage}

% Título de secciones en negrita y tamaño ajustado
\titleformat{\section}{\large\bfseries}{\thesection}{1em}{}

\begin{document}

% LOGO y encabezado
\begin{minipage}{0.2\textwidth}
    \includegraphics[width=\linewidth]{logo-usm.jpg} % Cambia logo.png por tu archivo real
\end{minipage}
\hfill
\begin{minipage}{0.75\textwidth}
    \centering
    \vspace{0.5cm}
    {\LARGE \textbf{Hito 1: SEMAT}}\\[0.4cm]
    {\large Universidad Tecnica Federico Santa Maria}\\
    {\large Asignatura: INF225 - Ingeniería de Software}\\
    {\large Fecha: \today}
\end{minipage}

\vspace{1.5cm}

% Participantes
\noindent
\textbf{Grupo 1:}\\
Matias Pajarito Catalan - \\
Pedro Miranda Miranda - \\
Aymara Rojas Arellano - \\
Andreu Lechuga Gonzalez - 202073595-6

\vspace{1cm}

\section*{Introducción}

El presente documento tiene como propósito analizar 
el desarrollo de nuestro proyecto de software utilizando 
el \textbf{framework SEMAT (Software Engineering Method 
and Theory)}. Este enfoque nos permite observar el 
progreso y madurez del proyecto desde distintas 
perspectivas, conocidas como "Alphas", facilitando una 
gestión más clara, estructurada y alineada con buenas 
prácticas de ingeniería de software.


\section*{Autoevaluacion SEMAT-Essence: Estados Actuales}

\section{1er Alpha: Requisitos}

\textbf{Estado Actual:} \textit{Acotado} \\
Los requisitos están definidos, pero falta validación con usuarios.

\vspace{0.2cm}

\textbf{Próximo Estado Deseado:} \textit{Coherente} \\
Los requisitos están alineados con las necesidades reales.

\vspace{0.5cm}

\textbf{Acciones Necesarias:}
\begin{itemize}
    \item Validacion con \textbf{Usuarios Reales:} Verificar que tanto 
    el boletín como el Software cumpla con las expectativas del cliente.
    \item Revision de \textbf{Requisitos:} Eliminar ambigüedades en los 
    requisitos actuales y definir criterios de aceptación nitidos y alcanzables.
\end{itemize}

\vspace{0.3cm}

\textbf{Justificación:} Se han definido los requisitos, pero falta validación de usuarios finales.


\section{2do Alpha: Sistema de Software}

\textbf{Estado Actual:} \textit{Demostrable} \\
El sistema es capaz de generar boletines, sin embargo aun
no es completamente estable ni esta correctamente optimizado
\vspace{0.2cm}

\textbf{Próximo Estado Deseado:} \textit{Usable} \\
El sistema debe ser capaz de generar boletines confiables 
de manera automatizada

\vspace{0.5cm}

\textbf{Acciones Siguientes:}
\begin{itemize}
    \item Optimizar el \textbf{Tiempo de Generacion de 
    Boletines}  para mejorar la escabilidad 
    \item Implementar \textbf{Mecanismos de Validacion de Contenido}
    para evitar la captura y uso de informacion irrelevante
    \item Realizar \textbf{Pruebas de Carga} para evaluar 
    rendimiento bajo alta demanda
\end{itemize}

\vspace{0.3cm}

\textbf{Justificación:} El sistema genera boletines, 
pero aún tiene limitaciones de estabilidad y optimización.

\end{document}